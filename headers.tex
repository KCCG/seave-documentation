%----------------------------------------------------------------------------------------
%	ABOUT
%----------------------------------------------------------------------------------------

\newpage

\section{About}

Seave is a web-based platform for genetic variant filtration, primarily for use in clinical genomics. It stores short variants in the form of single nucleotide polymorphisms (SNPs) and insertions/deletions (Indels). Long genetic variants such as copy number variants (CNV), structural variants (SV) and losses of heterozygosity (LoH) are stored in the form of genomic blocks of any size. Variants can be filtered using a variety of parameters and the results are annotated with a large number of external annotation databases.

The development of Seave began in January 2015 to provide the \href{https://garvan.org.au/kccg}{Kinghorn Centre for Clinical Genomics} (KCCG) at the \href{http://garvan.org.au}{Garvan Institute} a place to store genomic data and make it accessible to clinicians and researchers without bioinformatics backgrounds. It grew out of a simple front end to \GEMINI to include many advanced features. It has gone on to be used commercially within \href{https://genome.one}{Genome.One}, a spin-off from KCCG, and for a variety of germline and somatic research projects within and outside the \href{http://garvan.org.au}{Garvan Institute}. 

Seave was designed and developed by \href{https://www.vel.nz}{Dr.~Velimir Gayevskiy}. Dr.~Tony Roscioli contributed early advice for improving inheritance logic and usability for clinicians. Dr.~Mark Cowley has overseen development and contributed significantly to feature requests and code reviews.

%----------------------------------------------------------------------------------------
%	ACKNOWLEDGEMENTS
%----------------------------------------------------------------------------------------

\section{Acknowledgements}

We thank the Translational Genomics group (KCCG) and Genome.One for their comments and questions that lead to new features and bug fixes: Lisa Ewans, Eric Lee, Marie Wong, Kishore Kumar, Clare Puttick and André Minoche.

Seave would not be possible without the existence of \GEMINI, a free academic tool created by Dr.~Aaron Quinlan at the University of Utah. His work allowed us to rapidly prototype Seave and use GEMINI in any way we desire due to his generous MIT Licensing.